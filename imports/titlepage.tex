%!TEX root = ../document.tex

% center geometry on maketitle
\newgeometry{margin=1in,left=0.34\paperwidth}
\renewcommand\BackgroundPic{%
\put(310,-180){%
\parbox[b][\paperheight]{\paperwidth}{%
\vfill
\includegraphics[width=0.3\paperwidth,height=\paperheight,%
keepaspectratio]{imports/thi_logo.eps}
\vfill
}}}
\AddToShipoutPicture*{\BackgroundPic}

\begin{titlepage}
	\begin{tikzpicture}[remember picture,overlay]
		\fill[documentColor] ($(current page.south west)$) -- ($(current page.north west)$)  -- ($(current page.north west) + (0.33\paperwidth,0)$) -- ($(current page.south west) + (0.33\paperwidth,0)$)-- cycle ;
		%\path($(current page.south west) + (0.34\paperwidth,0)$) -- ($(current page.north west) + (0.34\paperwidth,0)$) node [midway, above, sloped, color=white] (TextNode) {\sffamily\fontsize{70}{60}\selectfont \MakeUppercase{\textls*[50]{computerforensik}}};
		\draw[color=black!40,line width=2mm]($(current page.south west) + (0.33\paperwidth,0)$) -- ($(current page.north west) + (0.33\paperwidth,0)$);
	\end{tikzpicture}

	\vspace{0.23\paperheight}
	\parbox[t]{\textwidth}{
		\sffamily\color{documentColor}\Huge{Netzwerkforensik: \\[0.2cm] Erkennung von SQL-Injections} \\[0.6cm]
		\sffamily\color{black!50}\Large{Stefan Braun} \\[0.3cm]
		\sffamily\color{black!50}\Large{11. Juni 2016} \\[0.3cm]
	}
\end{titlepage}

\clearpage


% tufte geomtery
\newgeometry{left=24.8mm,top=27.4mm,headsep=2\baselineskip,textwidth=127mm,marginparsep=8mm,marginparwidth=48mm,textheight=43\baselineskip,headheight=\baselineskip}
% marginnote requires to update the textwidth after newgeometry
\edef\marginnotetextwidth{\the\textwidth}